% !TEX root = TMV_Documentation.tex

\section{History}
\label{History}

Here is a list of the changes from version to version.  Whenever a change 
is not backward compatible, meaning that code using the previous version
might be broken, I mark the item with a $\times$ bullet rather than the usual $\bullet$
to indicate this.  
Also, the bulleted lists are not comprehensive.  
In most cases, new versions fix minor bugs that I find in the old version.  I only 
list the more significant changes.

\begin{description}
\item[Version 0.1]
The first matrix/vector library I wrote.  It wasn't very good, really. 
It had a lot of the functionality I needed, like mixing complex/real, 
SV decomposition, LU decomposition, etc.  But it wasn't at all fast for 
large matrices.  It didn't call BLAS or LAPACK, nor were the native routines
very well optimized.  Also, while it had vector views for rows and columns, 
it didn't have matrix views for things like transpose.  Nor did it have any
delayed arithmetic evaluation.  And there were no special matrices.

I didn't actually name this one 0.1 until I had what I called version 0.3.

\item[Version 0.2]
This was also not named version 0.2 until after the fact.  It had most of 
the current interface for regular Matrix and Vector operations.  I added
Upper/Lower TriMatrix and DiagMatrix.  It also had matrix views and
matrix composites to delay arithmetic evaluation.  The main problem was that
it was still slow.  I hadn't included any BLAS calls yet.  And while the 
internal routines at least used algorithms that used unit strides whenever 
possible, they didn't do any blocking or recursion, which are key for large
matrices.

\item[Version 0.3]
Finally, I actually named this one 0.3 at the time.
The big addition here was BLAS and LAPACK calls, which 
helped me to realize how slow my internal code really was (although I hadn't
updated them to block or recursive algorithms yet).
I also added BandMatrix.

\item[Version 0.4]
The new version number here was because I needed some added functionality for a
project I was working on.  It retrospect, it really only deserves a 0.01 increment,
since the changes weren't that big.  But, oh well.

\begin{itemize}

\item
Added QR\_Downdate.  (This was the main new functionality I needed.)

\item
Improved the numerical accuracy of the QRP decomposition.

\item
Added the possibility of not storing U,V for the SVD.

\item
Greatly improved the test suite, and consequently found and corrected a few bugs.

\item
Added some arithmetic functionality that had been missing (like \tt{m += L*U}).

\end{itemize}

\item[Version 0.5]
The new symmetric matrix classes precipitated a major version number update.
I also sped up a lot of the algorithms:

\begin{itemize}

\item
Added SymMatrix, HermMatrix, and all associated functionality.

\item
Added blocked versions of most of the algorithms, so the non-LAPACK
code runs a lot faster.

\item
Allowed for loose QRP decomposition.

\item
Added divideInPlace().

\end{itemize}

\item[Version 0.51]
Some minor improvements:

\begin{itemize}

\item
Sped up some functions like matrix addition and assignment by adding the 
LinearView method.

\item
Added QR\_Update, and improved the QR\_Downdate algorithm.

\item
Blocked some more algorithms like TriMatrix multiplication/division, so non-BLAS
code runs significantly faster (but still much slower than BLAS).

\end{itemize}

\item[Version 0.52]
The first ``public'' release!  And correspondingly, the first with documentation
and a web site.  A few other people had used previous versions, but since the
only documentation was my comments in the .h files, it wasn't all that user-friendly.

\begin{itemize}

\item
Added saveDiv() and related methods for division control.  Also changed the 
default behavior from saving the decomposition to not saving it.

\item
Added in-place versions of the algorithms for $S = L^\dagger L$ and $S = L L^\dagger$.

\end{itemize}

\item[Version 0.53]
By popular demand (well, a polite request by Fritz Stabenau, at least):

\begin{itemize}

\item
Added the Fortran-style indexing.

\end{itemize}

\item[Version 0.54]
Inspired by my to-do list, which I wrote for Version 0.52, I tackled a few of the 
items on the list and addressed some issues that people had been having
with compiling:

\begin{itemize}

\item[$\times$]
Changed from a rudimentary exception scheme (with just one class - 
\tt{tmv\_exception}) to the current more full-featured exception hierarchy.
Also added \tt{auto\_ptr} usage instead of bald pointers to make the 
exception throws memory-leak safe.

\item
Sped up SymLUDiv and SymSVDiv inverses.

\item
Added the possibility of compiling a small version of the library and test suite.

\item[$\times$]
Consolidated SymLUDiv and HermLUDiv classes into just SymLUDiv, which now checks
whether the matrix is hermitian automatically.  

\item
Reduced the number of operations that make temporary matrices when multiple
objects use the same storage. 

\item
Specialized Tridiagonal $\times$ Matrix calculation.

\item
Added ElementProd and AddElementProd functions for matrices.

\item
Added CLAPACK and ACML as known versions of LAPACK.  

\end{itemize}

\item[Version 0.60]
This revision merits a first-decimal-place increment, since I added a 
few big features.  I also registered it with SourceForge, 
which is a pretty big milestone as well.

\begin{itemize}

\item
Added \tt{SmallVector} and \tt{SmallMatrix} with all accompanying algorithms.

\item 
Added \tt{SymBandMatrix} and \tt{HermBandMatrix} with all accompanying algorithms.

\item
Made arithmetic between any two special matrices compatible, so long as 
the operation is allowed given their respective shapes.  

\item[$\times$]
Changed \tt{QR\_Downdate()} to throw an exception rather than return false
when it fails.

\item
Added the GPL License to the end of this documentation, and a copyright
and GPL snippet into each file.

\item[$\times$]
Changed the -D compiler options for changing which types get instantiated.

\item
Split up the library into \texttt{libtmv.a} and \texttt{libtmv\_symband.a}. 

\end{itemize}

\item[Version 0.61]
A number of updates mostly precipitated by feature requests by me in my own
use of the library, as well as some from a few other users.
I also did a complete systematic edit of the documentation which 
precipitated some more changes to make the UI a bit more intuitive.

\begin{itemize}

\item[$\times$]
Changed the default storage for the \tt{Matrix} class to \tt{ColMajor}
rather than \tt{RowMajor}.

\item[$\times$]
Changed a lot of \tt{size\_t} parameters to \tt{int}.

\item[$\times$]
Removed \tt{U.MakeUnitDiag}.

\item
Sped up matrix multiplication for non-blas implementations, including
openmp pragmas to allow for multiple threads on machines that support them.

\item
Changed a few things which prevented Microsoft Visual C++ from compiling successfully.  
Thanks to Andy Molloy
for spearheading this effort and doing the lion's share of the work to make
the code compatible with the VC++ compiler.

\item[$\times$]
Removed the optional index parameter to the non-method versions of 
MaxElement, etc.  

\item
Added an optional \tt{scale} parameter to \tt{m.NormSq(scale)}.

\item[$\times$]
Added the explicit decomposition routines.  I also
got rid of the \tt{SVU}, \tt{SVV} and \tt{SVS} options for \tt{m.divideUsing(...)}, since the point 
of these was to do the decomposition without calculation $U$ and/or $V$.  This is now done
more intuitively with the explicit decomposition routines.
I also added the (hermitian) eigenvalue/eigenvector routines which used to
require using the division accessors in non-intuitive ways to calculate.

\item 
Fixed a couple of places where underflow and overflow could cause problems.  

\item
Updated the native TMV code for the singular value decomposition and hermitian 
eigenvalue calculation to use the divide-and-conquer algorithm.

\item
Added \tt{m.logDet()} method.  

\item[$\times$]
Changed \tt{m.svd().getS()} to return a \tt{DiagMatrix} rather than a \tt{Vector}.

\end{itemize}

\item[Version 0.62]

This release contains a number of bug fixes and corrections of mistakes in the documentation.
I also significantly revamped the \tt{SmallMatrix} class to make it faster at the expense of a few fewer features.  

\begin{itemize}

\item
Corrected an error with \tt{m.divIsSet()}.

\item
Corrected some errors in the documentation -- Thanks to Jake VanderPlas for pointing these out.

\item
Improved the behavior of the division accessors, also at the suggestion of Jake VanderPlas. (Thanks again!)  They used to require that the division
object had already been set or they would give an error. Now the accessors (like \tt{m.svd()}) set
the division object appropriately if it is not already set. 

\item
Added the \tt{ListInit} method for initializing the values of a \tt{Vector} or \tt{Matrix}.  Thanks to Paul
Sarli for this suggestion.  

\item[$\times$]
Significantly changed the \tt{SmallMatrix} class.  See \S\ref{SmallMatrix}, on the \tt{SmallMatrix} class 
in the documentation
for the full details of how it works now.  But in brief, here are the main changes:
\begin{itemize}
\item \tt{SmallMatrix} and \tt{SmallVector} no longer inherit from \tt{GenMatrix} and \tt{GenVector}.
\item This allowed us to remove all virtual functions and the corresponding \tt{vtable}.
\item Improved the arithmetic so more of the routines correctly do the calculation inline to allow the
compiler more opportunity to optimize the calculation.
\item Added inlined division and determinants.
\item Got rid of the "Small" views. 
\end{itemize}

\item
Consolidated some of the old header files.  

\item[$\times$]
Removed some of the \tt{ViewOf} commands that were not very clear and
which have other, clearer ways of doing the same thing:
\begin{itemize}
\item
\tt{d = DiagMatrixViewOf(m)} should now be written\\
\tt{d = DiagMatrixViewOf(m.diag())}.
\item
\tt{U = UpperTriMatrixViewOf(m)} should now be written\\
\tt{U = m.upperTri()}.
\item
\tt{U = UpperTriMatrixViewOf(m,UnitDiag)} should now be written\\
\tt{U = m.upperTri(UnitDiag)}.
\item
\tt{U = UpperTriMatrixViewOf(U,UnitDiag)} should now be written\\
\tt{U = U.viewAsUnitDiag()}.
\end{itemize}

\item
Tracked down the problems I had been having with the LAPACK \tt{dstegr} and \tt{sstegr} 
functions.  TMV now checks for some known problems with the LAPACK implementation
of \tt{stegr} sometimes failing, and it calls \tt{stedc} if there was a problem.

\item
Tested the code for memory bugs with Paul Nettle's mmgr.h code.  There were only a couple of 
minor memory leaks, which were fixed.

\item
Fixed a problem with the OpenMP version of the code giving segmentation faults with pgCC.

\item
Added the SCons installation method to automatically discover what BLAS and LAPACK libraries 
are installed on your computer.  

\item
Added compatibility for generic Fortran versions of BLAS and LAPACK.

\item
Added the CMake installation method.  Thanks to Andy Molloy for providing this.  

\item
Added hyperlinks to the PDF documentation. Also, I added an index, complete with hyperlinks as well, that should make it easier to find a particular topic that you might need help with.

\end{itemize}

\item[Version 0.63]
The biggest thing in this version is the new lowercase syntax for the
methods, which was in response to feedback from a number of TMV users, 
who didn't like the old syntax.  Hopefully, I'm not responding to the 
minority here, but I do agree that the new syntax conforms better
to common C++ standards.  There are also a couple of bug fixes, and I've
started using Google Code for my main website.

\begin{itemize}

\item[$\times$]
Changed the names of the methods to start with a lowercase.
The free functions all still start with a capital letter.
Both use camelCase, with underscores after 2 or 3 letter abbreviations
such as LU or SV to help them stand out (since the camel case doesn't do it).
This seems to be a more common standard in the C++ community and is more
self-consistent than the style I had been using.

\item[$\times$]
Changed the syntax for list initialization to not use \tt{tmv::ListInit}.

\item
Improved the speed of the native (i.e. non-BLAS) matrix multiplication algorithm.

\item
Fixed a couple of bugs involving \tt{SmallMatrix}.  

\item
Disabled the cases where \tt{BlasRowMajor} was used in CBlas implementations, 
since I had problems
with it on a particular BLAS version, and I'm not sure if the error
is in my code or in that BLAS implementation.
So, I removed these pathways for now until I can figure out what might
be going on with it.  Most calls were with \tt{BlasColMajor} anyway, so this
doesn't affect very many calls.

\item
Fixed the return type of some methods that erroneously returned \tt{CStyle}
views when the source object was \tt{FortranStyle}.

\item
Added a way to access the version of TMV being used.
You can do this either in C++ code with the function \tt{TMV\_Version()}.
This returns a string that gives the version being used within your code.
There is also an executable, \tt{tmv-version}, that can be used to access
the TMV version that is installed at the moment.

\item
Added support for \tt{ups}, which is probably completely irrelevant
to the majority of users.  But if you use \tt{ups} for version control, you can
use the command \tt{scons install WITH\_UPS=true} to configure tmv in \tt{ups}.

\end{itemize}

\item[Version 0.64]
This update mostly addresses problems involving underflow and overflow.  This had 
been item 13 on the to-do list (\S\ref{Bugs}), and it was bumped to top priority from
a bug report where underflow problems were leading to an infinite loop in the 
singular value decomposition code.  So I added more matrices to the test suite -- 
specifically a matrix with an extremely large dynamic range of singular values
(which reproduced the behavior of the bug report), a matrix that is nearly zero,
and a matrix that is very close to overflow.  These tests uncovered quite a few bugs 
in the code.  Some from overflow and underflow issues of course, but also some
imprecision issues, and some just plain bugs that I hadn't uncovered before.  So 
I think this version should be much more stable for a wider range of input matrices.

There are also a bunch of miscellaneous feature additions as well -- most notably the \tt{Permutation} class.

\begin{itemize}

\item
Added a some very difficult matrices to the test suite, and found and fixed quite a few 
bugs as a result involving underflow, overflow, and loss of precision.  

\item 
Fixed a bug that \tt{U(i,i)} was invalid if \tt{U} is non-const and is \tt{UnitDiag} even 
if the resulting value is not used in a mutable context.  

\item 
Added a new \tt{Permutation} class to encapsulate permutation operations.  

\item[$\times$] 
The \tt{getP()} methods from the LU and QRP divider objects now return
a \tt{Permutation} rather than a \tt{const int*}.  

\item 
Added \tt{m.sumElements()} and \tt{m.sumAbsElements()}.

\item 
Added \tt{v.minAbs2Element()}, \tt{v.maxAbs2Element()} and \tt{m.maxAbs2Element()}.

\item 
Added \tt{m.unitUpperTri()} and \tt{m.unitLowerTri()}.

\item 
Added \tt{b.subBandMatrix(i1,i2,j1,j2)} to \tt{BandMatrix} and \tt{SymBandMatrix}.

\item 
Made \tt{Swap(m1,m2)} and \tt{Swap(v1,v2)} efficient when the arguments are both
complete matrices or vectors, rather than views.

\item 
Added \tt{cView()} and \tt{fView()} methods to switch indexing styles of a matrix or vector.

\item 
Made arithmetic with \tt{SmallVector} and \tt{SmallMatrix} somewhat more flexible in that
the composite objects now (again actually) derive from the normal \tt{GenVector} and
\tt{GenMatrix} classes.  

\item 
Added a version of \tt{MatrixViewOf} with arbitrary steps.

\item 
Added the three C preprocessor definitions: \tt{TMV\_MAJOR\_VERSION}, 
\tt{TMV\_MINOR\_VERSION},
and \tt{TMV\_VERSION\_AT\_LEAST(major,minor)} to help users customize
their code depending on the version of TMV that is installed on a particular
computer.  

\item 
Fixed a bug in BLAS versions of U*M and L*M when U or L is real and
M is complex.

\item 
Fixed a bug in \tt{m.det()} when \tt{m} is a \tt{SmallMatrix<T,1,1>}.

\item 
Fixed a bug in an (apparently) uncommon pathway of \tt{Rank2Update}.

\item 
Researched my \tt{QRDowndate} algorithm to see if it is in the literature already.
The only similar algorithm I could find was in Bojanczyk and Steinhardt (1991), 
a paper which seems to have been overlooked by the matrix algorithms community.
It's not quite the same as mine in detail, but it is based on the same basic idea.
I added this reference to the discussion in the documentation about my algorithm.

\end{itemize}

\item[Version 0.65]

This update primarily fixed some problems for integer matrices.  In addition to some compiler
errors for some combinations of SCons parameters, I also added the ability to calculate the 
determinant of integer matrices.  It had been documented (albeit somewhat poorly) that 
this would not work for integer matrices.  However, user Ronny Bergmann convinced me that
it would make sense to include the ability to produce accurate determinants for integer matrices.

\begin{itemize}

\item 
Made \tt{m.det()} produce accurate results if \tt{m} is a \tt{Matrix<int>} or 
\tt{Matrix<complex<int> >}.  Or in general if the underlying type is an integer type.

\item 
Added \tt{m.sumAbs2Elements()}.

\item 
Added \tt{m.addToAll(x)}.

\item 
Added \tt{m.unitUpperTri()} and \tt{m.unitLowerTri()} for symmetric/hermitian matrices.

\item 
Added the \tt{resize} methods to resize an existing vector or matrix, at the request of users Zhaoming Ma and Gary Bernstein.  

\item 
Added SCons option \tt{SHARED=true} to compile as a shared library rather than a static library.

\item 
Added the ability to turn off TMV debug statements with \tt{-DTMVNDEBUG}.

\item 
Skip compiling functions that are invalid for integer values, like the various decompositions 
and such.  

\item 
Fixed some compiling errors with \tt{INST_INT=true}, including some linking errors 
for several combinations of compile options.

\item 
Fixed a couple bugs in the allocation code.

\item 
Increased the maximum number of iterations within the divide-and-conquer SVD
algorithm from 20 to 100, since I found a case that required 27 iterations to converge.

\item 
Fixed some places where nan's in the input values could lead to
infinite loops in certain algorithms. 

\item 
Fixed a place in the SV decomposition which could (rarely) go into an infinite loop.

\item 
Made code more robust to broken BLAS distributions that incorrectly multiply the output matrix or vector by the \tt{beta} variable even if it is zero.  This is a problem if there are \tt{nan}'s present in the memory, since multiplies by zero doesn't set the value to zero in that case.  Now we always check for \tt{beta == 0} and zero out the memory if appropriate.

\end{itemize}

\item[Version 0.70]

The impetus for this release was mostly just the first change listed below
about the behavior of comma initialization for matrices.  But since I knew
this change would not be backwards compatible, and might require users
to edit their code, I decided to try to put all the non-backward-compatible changes that I've been thinking about introducing in the near future into this
release to try to do everything at once. 

\begin{itemize}

\item[$\times$] 
Changed the behavior of the \tt{m << 1, 2, 3...} initialization.  Now the items in the initialization list should be in row-major order regardless of the storage order of the matrix \tt{m}.  

\item[$\times$] 
Removed the constructors from either a C-style array or a \tt{std::vector}.  
To take the place of these constructors' functionality, I added iterators that iterate over the matrix in either row-major or column-major order, so the assignment can be done with the standard library's \tt{std::copy} function.
In addition, \tt{BandMatrix} can use \tt{m.diagmajor\_begin()} to iterate in diagonal-major order.

\item 
Added the ability to do comma initialization for some special matrix types.  These are also always listed in row-major order regardless of the actual storage order of the matrix in memory.  Also, only the values in the corresponding parts of the matrices should be listed in the initialization. 

I chose not to add similar initializers for symmetric or hermitian matrix types, because I think initialization of them is clearer in terms of only the upper or lower triangle part that is stored in memory.  

\item
Turned off error checking for the number of elements for a list 
initialization when TMV is in non-debugging mode (i.e. either \tt{NDEBUG}
or \tt{TMV\_NDEBUG} is defined).

\item[$\times$]
Changed all (non-\tt{Small}) matrix and vector types to only use 2 template
parameters.  The second parameter is now an integer that includes all
the information that was in the sometimes several parameters.  The 
values should be combined using the bitwise or operator (\tt{|}).  

Now you can leave out {\em any} parameters that you don't want to specify,
and the default value would be used.  And for the values you do specify, you can list them in any order now,
so you don't have to try to remember which parameter is supposed to be first.  
 
\item[$\times$]
Removed the functions \tt{read}, \tt{write} and \tt{writeCompact}, and replaced them with a more streamlined I/O interface.  See \S\ref{IOStyle} for more information.

\item[$\times$]
Removed the \tt{auto_ptr} read functions.  

\item
Added default constructors for the various \tt{Vector} and \tt{Matrix} classes.  These construct a zero-sized vector or matrix, so normally you would want to resize it at some point.  

\item[$\times$] Removed the \tt{ElementProd} and \tt{AddElementProd} functions.  The equivalent calculation is now done with a new operator called \tt{ElemProd(a,b)} which returns a composite object representing the element-wise product of two vectors or matrices. 

\item[$\times$] 
Removed the routines that returned pointers to a \tt{BaseMatrix} created with \tt{new}.  

\item[$\times$]
Changed the second template parameter of \tt{VIt} and \tt{CVIt} from either \tt{Step} or \tt{Unit} to the actual step size between elements, if known.  If the step size is not known at compile time, you can use the special value \tt{tmv::Unknown}, and the value will be determined at run time with a variable.  This allows the user more flexibility for optimizing code.  

\item 
Added a few more ``\tt{ViewOf}'' functions to allow direct views of memory as different kinds of matrix types directly specifying the step in each direction.  These are more flexible than the versions that merely specify a \tt{StorageType} to use.  For non-contiguous data, those weren't sufficient.  So now, you can view arbitrary data in memory as an \tt{UpperTriMatrixView}, \tt{LowerTriMatrixView},
\tt{DiagMatrixView}, \tt{BandMatrixView}, \tt{SymMatrixView}, or \tt{SymBandMatrixView} specifying the memory address of $m(0,0)$ and the step size in each direction.

\item[$\times$]
Changed the type of the matrix and vector size values from \tt{size\_t} to \tt{ptrdiff\_t}. 

\item[$\times$]
Changed all of the variables indexing offsets within the matrix (e.g. the arguments \tt{i} and \tt{j} of \tt{m(i,j)}) from \tt{int} to \tt{ptrdiff\_t} to avoid 32-bit \tt{int} overflows with matrices that are larger than 2 GBytes.  

\item[$\times$]
Changed the \tt{svd().getV()} method to \tt{svd().getVt()} to conform to the
usual definition of the SVD: $A = U S V^\dagger$.  The method \tt{getVt()} now refers to the matrix $V^\dagger$.  

\item
Added the ability of \tt{SymMatrix} SVD division to be done in place.
In addition to the memory used for the \tt{SymMatrix} values, it also uses 
the other half of the matrix that is normally not used for anything.

\item[$\times$]
Changed the mechanism for switching from the ``Loose QRP'' to the ``Strict QRP''
algorithm (and back).  Now you should set this with the function\\
\tt{tmv::UseStrictQRP();}\\
to start using the Strict QRP algorithm and\\
\tt{tmv::UseStrictQRP(false);}\\
to stop using it.  You can access the current state of the variable with\\
\tt{bool tmv::QRP_IsStrict()}

\item[$\times$]
Changed what ``\tt{const}'' in front of a \tt{VectorView},
\tt{MatrixView}, etc. means for the mutability of the view. 
Now a \tt{const VectorView} works the same as a 
\tt{ConstVectorView}.
That is, it is no longer able to modify the underlying data.

\item
Changed the definitions of the various decomposition functions
to accept either views (as before) or regular \tt{Matrix}, \tt{Vector},
etc. objects.  

\item[$\times$]
Removed the \tt{det} argument from the constructor for \tt{Permutation}.
Also, now the determinant is only calculated if you ask for it, rather 
than every time you create a \tt{Permutation}.

\item
Removed warnings related to the \tt{norm2} and \tt{makeInverseATA} methods.

\item 
Added a warning for when the matrix multiplication routine catches a \tt{bad\_alloc} exception and reverts to using a function that doesn't allocate memory.  Also added documentation for the warning that had been output when the divide-and-conquer SVD algorithm has trouble converging.

\item[$\times$]
Changed the default TMV warning behavior to be no output.  To turn on the warnings, the user needs to explicitly specify an \tt{ostream} with the \tt{tmv::WriteWarningsTo} function.

\item[$\times$]
Added a new compiler flag \tt{-DTMV\_EXTRA\_DEBUG} and moved all the debugging statements that are more than $O(1)$ time to use this guard, including initializations of matrices and vectors with \tt{888}.

\item[$\times$]
Changed the compiler flag \tt{-DTMVNDEBUG} to \tt{-DTMV\_NDEBUG}.

\item
Fixed an error in the \tt{TMV\_VERSION\_AT\_LEAST} marco.  The old version was actually backwards.  It mistakenly returned whether the library's version was equal to or {\em earlier} than the provided values, rather than equal or later.  The workaround if you need to check for versions before 0.70 is to use 
\begin{tmvcode}
#if TMV_MINOR_VERSION >= 70 
\end{tmvcode}
to correctly exclude v0.65 and earlier versions, rather than
\begin{tmvcode}
#if TMV_VERSION_AT_LEAST(0,70)
\end{tmvcode}
which will incorrectly allow version 0.65 and earlier through.  This workaround will work until I release a version 1.0 when the \tt{TMV\_MAJOR\_VERSION} value will change to 1, at which point hopefully nobody will need to be checking for versions earlier than v0.70 anymore.

\item
Fixed a bug where the symmetric SVD algorithm could go into an infinite loop (very rarely -- it required particularly strange properties in the input matrix).
 
\item
Made the Givens rotations a bit more robust to overflow and underflow.

\item
Fixed a bug that \tt{P*v} and \tt{P*m} and similar operators didn't work if 
\tt{v} or \tt{m} is \tt{Small}.

\item
Implemented a workaround for a bug in \tt{icpc} version 12.0.
It was getting an internal
error when compiling \tt{TMV\_SymCHDecompose.cpp}.  

\item 
Added support for the \tt{clang++} compiler.  

\item
Changed the default for the \tt{USE\_STEGR} SCons flag to \tt{false}.  

\item
Changed the default for the \tt{USE\_GEQP3} SCons flag to \tt{false}.

\item
Removed the \tt{make} and \tt{cmake} installation methods.  

\item
Switched the default value for the SCons option \tt{IMPORT\_ENV} to \tt{true}.

\item[$\times$]
Removed the functions and methods that had been deprecated in version 0.65.

\end{itemize}

\item[Version 0.71]

This release was the first one to be included in fink.
So on a Mac, if you use fink, you can now just type: 
\begin{verbatim}
fink install tmv
\end{verbatim}
to install TMV in \texttt{/sw/lib} and \texttt{/sw/include}.
However, the process of getting this to happen required a few changes to the SCons files.
I also learned a few SCons tricks that I've incorporated as well.

\begin{itemize}

\item
Added a new SCons parameter \tt{FINAL\_PREFIX}.

\item
Moved the installed \tt{tmv-link} file from \tt{PREFIX/share/tmv-link} to the fink-recommended location \tt{PREFIX/share/tmv/tmv-link}.

\item
Added documentation files to the installation process.  The files \tt{README},
\tt{CHANGELOG}, \tt{LICENSE}, and \tt{TMV\_Documentation} are now installed in
\tt{PREFIX/share/doc/tmv/}.

\item
Added a SCons flag \tt{IMPORT\_PREFIX} which specifies whether to include
\tt{PREFIX/include} and \tt{PREFIX/lib} in the search paths.

\item
Added a new feature that scons will now automatically try to determine 
how many cpus your system has and use that for the number of build jobs 
to run at once.  You can turn this off with either \tt{N\_BUILD\_THREADS} or \tt{scons -jN}.

\item
Changed the default value of \tt{SHARED} to true, and fixed up some features of the shared
library to conform to common standard practices.

\end{itemize}

\item[Version 0.72]

This release is mostly just some minor bug fixes and some changes in the 
installation defaults.  Probably the most significant change is that I have
switched the license from GPL to BSD.  

\begin{itemize}

\item Changed the license from GPL to BSD.

\item
Made Scons correctly detect when g++ is really clang++.

\item
Made the code compliant with the latest version of clang++. (\#9)

\item
Fixed some problems with the \texttt{install\_name} in the shared libraries.  
(\#3, \#7)

\item
Changed the default value of \texttt{INST\_INT} to \texttt{True}, so the \tt{Matrix<int>}
templates are instantiated now unless you specifically disable them.

\item
Added a default conversion from \tt{VarConjIter} to \tt{VIt}. (\#5)

\item 
Fixed a bug in the test suite about \tt{long double} I/O. (\#8)

\item
Updated the linking checks for MKL to work with the latest version (11.1).

\end{itemize}

\item[Version 0.73]

In addition to the below changes, the code is now hosted on
\href{\websitex}{GitHub}.

\begin{itemize}

\item
Removed an extraneous 16 bytes of memory usage from SmallVector. (\#12)

\item
Gives an error if PREFIX is the source directory. (\#13)

\item
Fixed compiler warnings from newer versions of clang++.

\item
Fixed tmv-version to work properly with git. (\#20)

\item
Added support for OpenMP for clang v3.7+.

\item
Improved automatic library selection for MKL. (\#15)

\item
Fixed broken \texttt{USE\_UNKNOWN\_VARS} feature. (\#16)

\item
No longer includes \texttt{\$PREFIX/include} in the include path by default.
Use scons \texttt{IMPORT\_PREFIX=true} if you want to include it. (\#17)

\item
Added \texttt{LINKFLAGS} as an explicit option. (\#21)

\item
Documented the sort order for eigenvectors returned by \texttt{Eigen}
and singular values returned \texttt{SV\_Decompose}. (\#14)

\end{itemize}

\item[Version 0.74]

This release just updated the code to be compatible with the new updates to the C++ standard.
The code now works properly using c++11, c++14, and the upcoming c++17 (as well as
continuing to work with c++03 and c++98). (\#22)

\item[Version 0.75]

This release makes the build compatible with SCons 3.0. (\#25)
It also fixes a minor bug in the signature of some operator= functions for
SmallVector and Vector.  (\#23)

\item[Version 0.76]

This release fixes some compiling issues with g++-14.  Specifically that it doesn't
allow \texttt{complex<int>} to be copied to \texttt{complex<long double>} without
explicit conversion.  There is no change to any functionality.

\end{description}
